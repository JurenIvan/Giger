\chapter{Arhitektura i dizajn sustava}
		
		\textbf{\textit{dio 1. revizije}}\\

		\textit{ Potrebno je opisati stil arhitekture te identificirati: podsustave, preslikavanje na radnu platformu, spremišta podataka, mrežne protokole, globalni upravljački tok i sklopovsko-programske zahtjeve. Po točkama razraditi i popratiti odgovarajućim skicama:}
	\begin{itemize}
		\item 	\textit{izbor arhitekture temeljem principa oblikovanja pokazanih na predavanjima (objasniti zašto ste baš odabrali takvu arhitekturu)}
		\item 	\textit{organizaciju sustava s najviše razine apstrakcije (npr. klijent-poslužitelj, baza podataka, datotečni sustav, grafičko sučelje)}
		\item 	\textit{organizaciju aplikacije (npr. slojevi frontend i backend, MVC arhitektura) }		
	\end{itemize}
	
	
	Da bi dugoročno uštjedjeli vrijeme, uložili smo dio vremena na konfiguiranje CI(continuous integration) i CD(continuous delivery) procesa.Za deploy aplikacije odabrali smo Heroku. Heroku je jedna od najpoznatijih platforma za deployanje aplikacija, a iz šume drugih ističe se svojom jednostavnošću.
	U usporedbi s AWS-ovim servisima koje smo razmatrali, heroku se sam brine oko instanci i arhitekture sustava na kojem se vrti naša aplikacija.Zbog ogranicenih resursa odlučili smo deployati aplikaciju na besplatnu instancu herokua.
	Besplatna instanca Herokua ima određena ograničenja, a najupecatljivije od njih je nacin na koji se pokrece projekt.
	Heroku sam prepozna kojeg je tipa projekt pa deployamo frontend i backend trebamo dvije instance.
	CD je integriran putem gitlabovih pipelinesa za koje je bilo potrebno napisati .gitlab-ci.yml file u kojem smo konfiguirali gitlabov pipeline. Gitlabov pipeline konfiguiran je tako da se na svaki komit u dev granu izgradi aplikacija, izvrte testovi i krene deploy na heroku.\\
	Da bi osigurali maksimalno vrijeme dostupnosti naše platforme, .yml datoteka također je konfiguirana tako da prilikom commita u master deploya aplikaciju na druge dvije instance.
	Ukupno imamo 4 pokrenute instance heroku, od koje su dvije backend, dvije frontend, dvije služe da development i dvije su za produkciju.U skladu s time, aplikacija ima 2 .properties datoteke. Jedna od njih je namjenjena lokalnom izvođenju aplikacije te sadrži postavke lokalne posgress baze, dok je druga konfiguirana tako postavke čita iz varijabli okruženja. Varijalbe okruženja postavljene su na Heroku tako da čak niti pristupom u git repozitorij vanjski korisnik nemože doći do akreditiva(credentials) kojima bi mogao pristupiti bazi.
	
	Prednost ostvarena automatiziranjem deployment procesa jest povećanje vjerojatnosti uspješnosti deploya te povećanje udjela dostupnosti aplikacije zbog dva para instanci servera.
	
	Uvidjevši prednosti korištenja continuous deploymenta, primjenili smo to znanje i na kompjaliranje dokumentacije. 
	Unutar repozitorija, osim pipelinea za deploy, postoji pipeline za automatsko kompjaliranje dokumentacije čiji je rezultat .pdf dokument.\\
	
	Potencijalni prostor za napredak bio bi korištenje Docker tehnologije tako da prilikom pokretanja aplikacije korisnik ne mora imati instaliranu posgres bazu već ju pokrene u docker kontejneru.\\

	Od mogućih arhitektura sustava, za svoj projekt smo odabrali objektno usmjerenu arhitekturu. Tu arhitekturu smo odabrali zato što se koristi u industriji te je defacto standard razvoja programskih rješenja. Osim toga, ona je fleksibilna, omogućuje recikliranje koda, te logički razdjeljuje sustav na više cjelina, što je bitno s obzirom da više ljudi radi na implementaciji aplikacije. Zahvaljujući modularnosti programskog rješenja, greške su lako ispravljive, a nove mogućnosti dodaju se lako u timu.\\

	Odlučili smo se za web aplikaciju, koja je prilagođena mobilni uređajima, obzirom da glazbenici, a time i bendovi nemaju uvijek pristup računalu, a ne želimo da je korisnik ograničen samo na mobilne uređaje.\\

	Arhitekturu sustava možemo podijeliti na tri podsustava:
		\begin{itemize}
			\item Web preglednik
			\item Web poslužitelj
    			\item Web aplikacija
			\item Baza podataka
		\end{itemize}


	Korisnik (javnost, glazbenik, bend, administrator) pristupa web aplikaciji uz pomoć svog web preglednika, s time da se u sredini nalazi web poslužitelj. Na njemu se nalazi aplikacija koju on pokreće, te uz pomoć protokola komunicira s korisnicima.\\

	Klijentski(frontend) dio aplikacije omogućuje da korisnik korištenjem sučelja može pristupiti serveru(backend) aplikacije. Ovisno o tome što korisnik hoće, taj server ima mogućnost spajanja na bazu podataka kako bi korisniku prikazao informacije.\\

	Backend je napisan u Javi, a kao razvojni okvir koristimo Java Spring Boot. Dodani su projekti Spring Data kako bi backend mogao lako komunicirati s bazom, Spring Web MVC za rukovanje sa request-ima te Spring Security kako bi zaštitili aplikaciju od vanjskih napada. \\

	\begin{figure}[H]
		\begin{center}
			\includegraphics[width=10cm]{slike/mvc.PNG}
		\end{center}
		\caption{Pojednostavljeni prikaz MVC-a}
		\label{fig:mvc}
	\end{figure}

	Za frontend koristimo React. On je moderni i jednostavan framework koji koristi HTML, CSS, JSX i JavaScript uz pomoć kojeg smo napravili sučelje za našu aplikaciju. Uz pomoć React-a možemo lagano komunicirati s backendom koristeći REST.\\




		\section{Baza podataka}
			
			\textbf{\textit{dio 1. revizije}}\\
			
		\textit{Potrebno je opisati koju vrstu i implementaciju baze podataka ste odabrali, glavne komponente od kojih se sastoji i slično.}
		
		Za potrebe razvoja \textit{Gigera} koristit će se objektno relacijsko mapiranje. To je metoda koja se koristi u objektno-orijentiranim jezicima te se na taj način stvara virtualna objektna baza podataka. Za implementaciju baze podataka odabrali smo Postgres, zbog generalno pozitivnog iskustva u korištenju te implementacije baze podataka u dosadašnjem fakultetskom obrazovanju. Bitno je naglasiti da na osobnim računalima u svrhu razvijanja aplikacije koristimo istu implementaciju baze kao i na web poslužitelju kako bi minimizirali neočekivano ponašanje.
		
		Baza podataka sastoji se od sljedećih entiteta:
		
		\begin{packed_item}
		\item Comment
		\item Message
		\item Conversation
		\item Conversation\_user
		\item System\_person
		\item People
		\item System\_person\_roles
		\item Organizer
		\item Conversation\_band
		\item Band
		\item Gig\_type
		\item Band\_occasions
		\item Occasion
		\item Musician\_occasions
		\item Band\_invited\_back\_up\_members
		\item Band\_invited
		\item Band\_back\_up\_members
		\item Musician\_bands
		\item Post
		\item Musician
		\item Instruments
		\item Instrument
		\item Musician\_gig\_history
		\item Gig
		\item Review\_gig
		\item Review
		
	\end{packed_item}
	
	
	\subsection{Opis tablica}
	
	\begin{longtabu} to \textwidth {|X[6, l+3]|X[6, l]|X[20, l]|}
		
		\hline \multicolumn{3}{|c|}{\textbf{Comment}}	 \\[3pt] \hline
		\endfirsthead
		
		\hline 
		\endlastfoot
		
		\textbf{id} & BIGINT	&  	jedinstveni identifikator komentara 	\\ \hline
		content & VARCHAR & sadržaj komentara \\ \hline
		posted\_on & TIMESTAMP & datum i vrijeme objave komentara \\ \hline	
		\textit{author\_id} & BIGINT & jedinstveni identifikator autora komentara \\ \hline
		\textit{fk\_post} & BIGINT & jedinstveni identifikator komentara \\ \hline
		
	\end{longtabu}
	
	\textit{\bf Message}
	\textit{Ovaj entitet sadrži informacije o poruci. Sadrži atribute: id poruke, sadržaj poruke, vrijeme kada je poruka poslana, id pošiljatelja i id razgovora. Ovaj entitet je u \emph{Many-to-Many} vezi  sa entitetom Message\_seen te je u \emph{Many-to-many} vezi sa entitetima Conversation i User.}
	\begin{longtabu} to \textwidth {|X[6, l+3]|X[6, l]|X[20, l]|}
		
		\hline \multicolumn{3}{|c|}{\textbf{Message}}	 \\[3pt] \hline
		\endfirsthead
		
		\hline
		\endlastfoot
		
		\textbf{id} & BIGINT	&  	jedinstveni identifikator poruke 	\\ \hline
		content	& VARCHAR & sadržaj poruke	\\ \hline
		sent\_time & TIMESTAMP & vrijeme kada je poruka poslana \\ \hline
		\textit{fk\_sender} & BIGINT & jedinstveni identifikator pošiljatelja \\ \hline
		\textit{fk\_sender\_band} & BIGINT & jedinstveni identifikator benda pošiljatelja \\ \hline
		\textit{fk\_converation} & BIGINT & jedinstveni identifikator razgovora \\ \hline
		
	\end{longtabu}
	
	\textit{\bf Conversation}
	\textit{Ovaj entitet sadrži informacije o razgovoru. Sadrži atribute: id razgovora i ime razgovora. Ovaj entitet je u \emph{One-to-Many} vezi  sa entitetima:Message i Conversation\_user.}
	\begin{longtabu} to \textwidth {|X[6, l+3]|X[6, l]|X[20, l]|}
		
		\hline \multicolumn{3}{|c|}{\textbf{Conversation}}	 \\[3pt] \hline
		\endfirsthead
		
		\hline
		\endlastfoot
		
		\textbf{id} & BIGINT	&  	jedinstveni identifikator razgovora 	\\ \hline
		\textit{fk\_band} & BIGINT & jedinstveni identifikator benda \\ \hline
		picture\_url & VARCHAR & url slike razgovora \\ \hline
		title	& VARCHAR &  naziv razgovora	\\ \hline
		
	\end{longtabu}
	
	\textit{\bf Conversation\_user}
	\textit{Ovaj entitet sadrži informacije o korisniku u razgovoru. Sadrži atribute: id korisnika i id razgovora. Ovaj entitet je u \emph{Many-to-One} vezi  sa entitetima:User i Conversation.}
	\begin{longtabu} to \textwidth {|X[6, l+3]|X[6, l]|X[20, l]|}
		
		\hline \multicolumn{3}{|c|}{\textbf{Conversation\_user}}	 \\[3pt] \hline
		\endfirsthead
		
		\hline
		\endlastfoot
		
		\textit{fk\_user} & BIGINT	&  	jedinstveni identifikator korisnika	\\ \hline
		\textit{fk\_conversation}	& BIGINT &  jedinstveni identifikator razgovora	\\ \hline
		
	\end{longtabu}
	
	
	\begin{longtabu} to \textwidth {|X[6, l+3]|X[6, l]|X[20, l]|}
		
		\hline \multicolumn{3}{|c|}{\textbf{System\_person}}	 \\[3pt] \hline
		\endfirsthead
		
		\hline
		\endlastfoot
		
		\textbf{id} & BIGINT	&  	jedinstveni identifikator sustavskih podataka o korisniku	\\ \hline
		email & VARCHAR & email adresa osobe \\ \hline
		locked & BOOLEAN & korisnik ima zabranu korištenja aplikacije ili ne \\ \hline
		password\_hash & VARCHAR & hash lozinke osobe \\ \hline
		verified & BOOLEAN & email adresa potvrđena ili ne \\ \hline
		
	\end{longtabu}
	
	\begin{longtabu} to \textwidth {|X[6, l+3]|X[6, l]|X[20, l]|}
		
		\hline \multicolumn{3}{|c|}{\textbf{Person}}	 \\[3pt] \hline
		\endfirsthead
		
		\hline
		\endlastfoot
		
		\textbf{id} & BIGINT	&  	jedinstveni identifikator korisnika	\\ \hline
		phone\_number & VARCHAR & telefonski broj korisnika \\ \hline
		picture\_url & VARCHAR & url slike korisnika \\ \hline
		username & VARCHAR & korisničko ime korisnika
		
	\end{longtabu}
	
	\begin{longtabu} to \textwidth {|X[6, l+3]|X[6, l]|X[20, l]|}
		
		\hline \multicolumn{3}{|c|}{\textbf{System\_person\_roles}}	 \\[3pt] \hline
		\endfirsthead
		
		\hline
		\endlastfoot
		
		\textbf{system\_person\_id} & BIGINT	&  	jedinstveni identifikator sustavskih podataka o korisniku	\\ \hline
		roles & INT & Uloga korisnika \\ \hline
		
		
	\end{longtabu}
	
	\textit{\bf Organizer}
	\textit{Ovaj entitet sadrži informacije za organizatora. Sadrži atribute: id organizatora te ime organizatora. Ovaj entitet je u \emph{One-to-Many} vezi  sa entitetima: Review\_organizer, Gig i User}
	\begin{longtabu} to \textwidth {|X[6, l+3]|X[6, l]|X[20, l]|}
		
		\hline \multicolumn{3}{|c|}{\textbf{Organizer}}	 \\[3pt] \hline
		\endfirsthead
		
		\hline
		\endlastfoot
		
		\textbf{id} & BIGINT	&  	jedinstveni identifikator organizatora 	\\ \hline
		manager\_name	& VARCHAR &  ime organizatora	\\ \hline
		
	\end{longtabu}
	
	\begin{longtabu} to \textwidth {|X[6, l+3]|X[6, l]|X[20, l]|}
		
		\hline \multicolumn{3}{|c|}{\textbf{Conversation\_band}}	 \\[3pt] \hline
		\endfirsthead
		
		\hline
		\endlastfoot
		
		\textit{fk\_band} & BIGINT	&  	jedinstveni identifikator benda	\\ \hline
		\textit{fk\_conversation}	& BIGINT &  jedinstveni identifikator razgovora	\\ \hline
		
	\end{longtabu}
	
	\begin{longtabu} to \textwidth {|X[6, l+3]|X[6, l]|X[20, l]|}
		
		\hline \multicolumn{3}{|c|}{\textbf{Band}}	 \\[3pt] \hline
		\endfirsthead
		
		\hline 
		\endlastfoot
		
		\textbf{id} & BIGINT	&  	jedinstveni identifikator benda 	\\ \hline
		bio & VARCHAR & opis benda \\ \hline
		formed\_date & DATE & datum osnutka benda \\ \hline
		address & VARCHAR & adresa benda \\ \hline
		extra\_description & VARCHAR & dodatak opis benda \\ \hline
		x & DOUBLE & x koordinata lokacije \\ \hline
		y & DOUBLE & y koordinata lokacije \\ \hline
		max\_distance & DOUBLE & najveća udaljenost koju bend želi prijeći zbog gaže \\ \hline
		name & VARCHAR & ime benda \\ \hline
		picture\_url & VARCHAR & url slike benda \\ \hline
		\textit{leader\_id}	& BIGINT &  jedinstveni identifikator voditelja benda	\\ \hline 	
		
	\end{longtabu}
	
	\begin{longtabu} to \textwidth {|X[6, l+3]|X[6, l]|X[20, l]|}
		
		\hline \multicolumn{3}{|c|}{\textbf{Gig\_type}}	 \\[3pt] \hline
		\endfirsthead
		
		\hline 
		\endlastfoot
		
		\textbf{gig} &  BIGINT	&  	jedinstveni identifikator vrste nastupa 	\\ \hline
		gig\_type	& VARCHAR &  vrsta nastupa	\\ \hline 		
		
	\end{longtabu}
	
	\begin{longtabu} to \textwidth {|X[6, l+3]|X[6, l]|X[20, l]|}
		
		\hline \multicolumn{3}{|c|}{\textbf{Band\_occasions}}	 \\[3pt] \hline
		\endfirsthead
		
		\hline 
		\endlastfoot
		
		\textit{occasion\_id} &  BIGINT	&  	jedinstveni identifikator događaja 	\\ \hline
		\textit{band\_id} &  BIGINT	&  	jedinstveni identifikator benda koji sudjeluje na događaju 	\\ \hline 		
		
	\end{longtabu}
	
	\begin{longtabu} to \textwidth {|X[6, l+3]|X[6, l]|X[20, l]|}
		
		\hline \multicolumn{3}{|c|}{\textbf{Occasion}}	 \\[3pt] \hline
		\endfirsthead
		
		\hline 
		\endlastfoot
		
		\textbf{id} &  BIGINT	&  	jedinstveni identifikator događaja 	\\ \hline
		description & VARCHAR & opis događaja \\ \hline
		local\_date & DATE & datum održavanja događaja \\ \hline
		personal\_occasion & BOOLEAN & privatan događaj ili ne \\ \hline
		\textit{band\_id} & BIGINT & jedinstveni identifikator benda koji sudjeluje na događaju
		
		
	\end{longtabu}
	
	\begin{longtabu} to \textwidth {|X[6, l+3]|X[6, l]|X[20, l]|}
		
		\hline \multicolumn{3}{|c|}{\textbf{Musician\_occasions}}	 \\[3pt] \hline
		\endfirsthead
		
		\hline 
		\endlastfoot
		
		\textit{musician\_id} &  BIGINT	&  	jedinstveni identifikator glazbenika 	\\ \hline
		\textit{occasions\_id} &  BIGINT	&  	jedinstveni identifikator događaja	\\ \hline
		
		
	\end{longtabu}
	
	\begin{longtabu} to \textwidth {|X[6, l+8]|X[6, l]|X[20, l]|}
		
		\hline \multicolumn{3}{|c|}{\textbf{Band\_invited\_back\_up\_members}}	 \\[3pt] \hline
		\endfirsthead
		
		\hline 
		\endlastfoot
		
		\textit{band\_id} &  BIGINT	&  	jedinstveni identifikator benda 	\\ \hline
		\textit{invited\_back\_up\_members\_id} &  BIGINT	&  	jedinstveni identifikator glazbenika pozvanih u bend kao rezerva	\\ \hline
		
		
	\end{longtabu}
	
	\begin{longtabu} to \textwidth {|X[6, l+3]|X[6, l]|X[20, l]|}
		
		\hline \multicolumn{3}{|c|}{\textbf{Band\_invited}}	 \\[3pt] \hline
		\endfirsthead
		
		\hline 
		\endlastfoot
		
		\textit{band\_id} &  BIGINT	&  	jedinstveni identifikator benda 	\\ \hline
		\textit{invited\_id} &  BIGINT	&  	jedinstveni identifikator glazbenika pozvanih u bend	\\ \hline
		
		
	\end{longtabu}
	
	\begin{longtabu} to \textwidth {|X[6, l+8]|X[6, l]|X[20, l]|}
		
		\hline \multicolumn{3}{|c|}{\textbf{Band\_back\_up\_members}}	 \\[3pt] \hline
		\endfirsthead
		
		\hline 
		\endlastfoot
		
		\textit{band\_id} &  BIGINT	&  	jedinstveni identifikator benda 	\\ \hline
		\textit{invited\_back\_up\_members\_id} &  BIGINT	&  	jedinstveni identifikator glazbenika koji su rezervni članovi	\\ \hline
		
		
	\end{longtabu}
	
	\begin{longtabu} to \textwidth {|X[6, l+3]|X[6, l]|X[20, l]|}
		
		\hline \multicolumn{3}{|c|}{\textbf{Musician\_bands}}	 \\[3pt] \hline
		\endfirsthead
		
		\hline 
		\endlastfoot
		
		\textit{fk\_musician} & BIGINT	&  	jedinstveni identifikator glazbenika 	\\ \hline
		\textit{fk\_band}	& BIGINT &  jedinstveni identifikator benda	\\ \hline 		
		
	\end{longtabu}
	
	\begin{longtabu} to \textwidth {|X[6, l+3]|X[6, l]|X[20, l]|}
		
		\hline \multicolumn{3}{|c|}{\textbf{Post}}	 \\[3pt] \hline
		\endfirsthead
		
		\hline 
		\endlastfoot
		
		\textbf{id} & BIGINT	&  	jedinstveni identifikator objave 	\\ \hline
		content & VARCHAR & sadržaj objave \\ \hline
		published\_on & TIMESTAMP & datum i vrijeme objave objave \\ \hline	
		\textit{fk\_band} & BIGINT & jedinstveni identifikator benda \\ \hline
		\textit{fk\_user} & BIGINT & jedinstveni identifikator korisnika koji je napisao objavu \\ \hline
		
	\end{longtabu}
	
	\begin{longtabu} to \textwidth {|X[6, l+3]|X[6, l]|X[20, l]|}
		
		\hline \multicolumn{3}{|c|}{\textbf{Musician}}	 \\[3pt] \hline
		\endfirsthead
		
		\hline 
		\endlastfoot
		
		bio	& VARCHAR &  opis glazbenika	\\ \hline 
		public\_calendar & BOOLEAN & kalendar glazbenika javan ili ne \\ \hline
		\textbf{id} & BIGINT	&  	jedinstveni identifikator glazbenika 	\\ \hline		
		
	\end{longtabu}
	
	\begin{longtabu} to \textwidth {|X[6, l+3]|X[6, l]|X[20, l]|}
		
		\hline \multicolumn{3}{|c|}{\textbf{Instruments}}	 \\[3pt] \hline
		\endfirsthead
		
		\hline 
		\endlastfoot
		
		\textit{instruments\_id} & BIGINT & jedinstveni identifikator instrumenta \\ \hline
		\textbf{musician} & BIGINT	&  	jedinstveni identifikator glazbenika	\\ \hline
		
		
	\end{longtabu}
	
	\begin{longtabu} to \textwidth {|X[6, l+3]|X[6, l]|X[20, l]|}
		
		\hline \multicolumn{3}{|c|}{\textbf{Instrument}}	 \\[3pt] \hline
		\endfirsthead
		
		\hline 
		\endlastfoot
		
		\textbf{id} & BIGINT & jedinstveni identifikator instrumenta \\ \hline
		name & VARCHAR & ime instrumenta \\ \hline
		type & INT & vrsta instrumenta \\ \hline
		
		
	\end{longtabu}
	
	\begin{longtabu} to \textwidth {|X[6, l+3]|X[6, l]|X[20, l]|}
		
		\hline \multicolumn{3}{|c|}{\textbf{Musician\_gig\_history}}	 \\[3pt] \hline
		\endfirsthead
		
		\hline 
		\endlastfoot
		
		\textit{fk\_musician} & BIGINT & jedinstveni identifikator glazbenika \\ \hline
		\textit{fk\_gig} & BIGINT & jedinstveni identifikator nastupa \\ \hline
		
		
		
	\end{longtabu}
	
	\begin{longtabu} to \textwidth {|X[6, l+3]|X[6, l]|X[20, l]|}
		
		\hline \multicolumn{3}{|c|}{\textbf{Gig}}	 \\[3pt] \hline
		\endfirsthead
		
		\hline 
		\endlastfoot
		
		\textbf{id} & BIGINT	&  	jedinstveni identifikator nastupa 	\\ \hline
		date\_time & TIMESTAMP & datum i vrijeme održavanja nastupa \\ \hline
		description & VARCHAR & opis nastupa \\ \hline
		expected\_duration & VARCHAR & očekivano trajanje nastupa \\ \hline
		final\_deal\_achieved & BOOLEAN & dogovor postignut ili ne \\ \hline
		\textit{gig\_type} & INT & vrsta nastupa \\ \hline
		address & VARCHAR & adresa održavanja nastupa \\ \hline
		extra\_description & VARCHAR & dodatan opis nastupa \\ \hline
		x & DOUBLE & x koordinata lokacije \\ \hline
		y & DOUBLE & y koordinata lokacije \\ \hline
		private\_gig & BOOLEAN & nastupa privatan ili ne \\ \hline
		proposed\_price & INT & preporučena cijena ulaznice \\ \hline
		\textit{organizer\_id}	& BIGINT &  jedinstveni identifikator organizatora	\\ \hline 	
		\textit{bend\_id}	& BIGINT &  jedinstveni identifikator benda	\\ \hline 	
		
	\end{longtabu}
	
	\textit{\bf Review\_gig}
	\textit{Ovaj entitet sadrži informacije za recenziju nastupa. Sadrži atribute: id nastupa i id recenzije. Ovaj entitet je u \emph{Many-to-One} vezi  sa entitetima:Gig i Review.}
	\begin{longtabu} to \textwidth {|X[6, l+3]|X[6, l]|X[20, l]|}
		
		\hline \multicolumn{3}{|c|}{\textbf{Review\_gig}}	 \\[3pt] \hline
		\endfirsthead
		
		\hline 
		\endlastfoot
		
		\textit{fk\_gig} & BIGINT	&  	jedinstveni identifikator nastupa 	\\ \hline
		\textit{fk\_review}	& BIGINT &  jedinstveni identifikator recenzije	\\ \hline 		
		
	\end{longtabu}
	
	\textit{\bf Review}
	\textit{Ovaj entitet sadrži informacije za recenziju. Sadrži atribute: id recenzije, sadržaj recenzije, ocjenu te id autora. Ovaj entitet je u \emph{One-to-Many} vezi  sa entitetima: Review\_band, Review\_gig, Review\_organizer i Review\_musician.}
	
	\begin{longtabu} to \textwidth {|X[6, l+14]|X[6, l+2]|X[20, l]|}
		
		\hline \multicolumn{3}{|c|}{\textbf{Review}}	 \\[3pt] \hline
		\endfirsthead
		
		\hline
		\endlastfoot
		
		\textbf{id} & BIGINT	&  	jedinstveni identifikator recenzije 	\\ \hline
		content\_of\_review\_for\_band	& VARCHAR &  sadržaj komentara benda	\\ \hline
		content\_of\_review\_for\_organizer	& VARCHAR &  sadržaj komentara organizatora	\\ \hline
		created & TIMESTAMP & vrijeme objave komentara \\ \hline
		grade\_band & INT & ocjena benda 1-5  \\ \hline
		grade\_organizer & INT & ocjena organizatora 1-5  \\ \hline
		\textit{author\_id} & BIGINT	& jedinstveni identifikator korisnika koji je autor recenzije	\\ \hline
		
		
	\end{longtabu}
	

		
	

			
			\subsection{Dijagram baze podataka}
			
			\begin{figure}[H]
			\begin{center}
				\includegraphics[width=17cm]{slike/ERModel.PNG}
			\end{center}
			\caption{Dijagram baze podataka}
			\label{fig:dijagramBaze}
		\end{figure}
			
			
			
		\section{Dijagram razreda}
		
			\textit{Potrebno je priložiti dijagram razreda s pripadajućim opisom. Zbog preglednosti je moguće dijagram razlomiti na više njih, ali moraju biti grupirani prema sličnim razinama apstrakcije i srodnim funkcionalnostima.}\\
			
			\textbf{\textit{dio 1. revizije}}\\
			
			\textit{Prilikom prve predaje projekta, potrebno je priložiti potpuno razrađen dijagram razreda vezan uz \textbf{generičku funkcionalnost} sustava. Ostale funkcionalnosti trebaju biti idejno razrađene u dijagramu sa sljedećim komponentama: nazivi razreda, nazivi metoda i vrste pristupa metodama (npr. javni, zaštićeni), nazivi atributa razreda, veze i odnosi između razreda.}\\
			
			\textbf{\textit{dio 2. revizije}}\\			
			
			\textit{Prilikom druge predaje projekta dijagram razreda i opisi moraju odgovarati stvarnom stanju implementacije}
			
			
			
			\eject
		
		\section{Dijagram stanja}
			
			
			\textbf{\textit{dio 2. revizije}}\\
			
			\textit{Potrebno je priložiti dijagram stanja i opisati ga. Dovoljan je jedan dijagram stanja koji prikazuje \textbf{značajan dio funkcionalnosti} sustava. Na primjer, stanja korisničkog sučelja i tijek korištenja neke ključne funkcionalnosti jesu značajan dio sustava, a registracija i prijava nisu. }
			
			
			\eject 
		
		\section{Dijagram aktivnosti}
			
			\textbf{\textit{dio 2. revizije}}\\
			
			 \textit{Potrebno je priložiti dijagram aktivnosti s pripadajućim opisom. Dijagram aktivnosti treba prikazivati značajan dio sustava.}
			
			\eject
		\section{Dijagram komponenti}
		
			\textbf{\textit{dio 2. revizije}}\\
		
			 \textit{Potrebno je priložiti dijagram komponenti s pripadajućim opisom. Dijagram komponenti treba prikazivati strukturu cijele aplikacije.}