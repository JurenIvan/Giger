\chapter{Implementacija i korisničko sučelje}
		
		
		\section{Korištene tehnologije i alati}
		
			\textbf{\textit{dio 2. revizije}} \\
			
			 \textit{Detaljno navesti sve tehnologije i alate koji su primijenjeni pri izradi dokumentacije i aplikacije. Ukratko ih opisati, te navesti njihovo značenje i mjesto primjene. Za svaki navedeni alat i tehnologiju je potrebno \textbf{navesti internet poveznicu} gdje se mogu preuzeti ili više saznati o njima}.
			
			
			\eject 
		
	
		\section{Ispitivanje programskog rješenja}
	
			
			\subsection{Ispitivanje komponenti}
			\textit{Potrebno je provesti ispitivanje jedinica (engl. unit testing) nad razredima koji implementiraju temeljne funkcionalnosti. Razraditi \textbf{minimalno 6 ispitnih slučajeva} u kojima će se ispitati redovni slučajevi, rubni uvjeti te izazivanje pogreške (engl. exception throwing). Poželjno je stvoriti i ispitni slučaj koji koristi funkcionalnosti koje nisu implementirane. Potrebno je priložiti izvorni kôd svih ispitnih slučajeva te prikaz rezultata izvođenja ispita u razvojnom okruženju (prolaz/pad ispita). }
			
			
			
			Da bismo testirali backend, napravili smo dvije vrste testova. JUnit testove za testiranje servisa te integracijske testove koje je najlakše opisati kao automatizirane Postman zahtjeve.
			
			Servisi su testirani na način da ih gledamo kao crne kutije koje za određeni ulaz trebaju odraditi određene akcije ili vratiti određene objekte. Nomenklatura testovi servisa (BandServiceTest) slijedi pravilo given\_when\_then. Naprimjer, ako testiramo metodu naziva createMusician, pretpostavljajuci da on još ne postoji i očekujući da se kreira glazbenik u bazi, naziv testa bio bi 
			noSuchMusician\_
			createMusician\_createAndPersistateMusician().
			
			Pomoću Mockito frameworka u testovima mockamo (oponašamo) ulaze te na taj način kontroliramo ulaze i okolinu metode koju testiramo. Drugim riječima, kada radimo test za neku komponentu ili servis, pretpostavljamo da su svi ulazi dobri i ne zanima nas utjecaj naše komponente na drugu komponentu, već samo direktni izlaz. Na taj način dobivamo testove koji su međusobno nepovezani i koji definiraju željeno ponašanje aplikacije. Ukoliko u daljnjem razvoju neki od razvojnih programera promijeni neko ponašanje koje ima utjecaj na druge komponente, pravilno napisani testovi trebali bi pasti i upozoriti ga da će se njegova promjena propagirati dublje u aplikaciju. Testovi koji su pisani ciljano na pojedine komponente sustava u kontroliranim uvjetima prilikom izvođenja precizno ukazuju na vjerojatan izvor pogreške. Na primjer, ako promijenimo implementaciju kreiranja glazbenika te on u trenutku kreiranja ne dobije id, samo testovi koji provjeravaju parametre nakon inicijalizacije bi trebali pasti, a ne svi testovi koji se u nekom trenutku pozivaju na tu funkcionalnost.
			
			Svaki napisan test odijeljen je u tri cjeline, a to su Arrange, Act, Assert. U prvom dijelu uređujemo i mockamo ulaze u testirajuću komponentu, na prije opisan način. U drugom dijelu testa poziva se akcija ili niz akcija čije djelovanje želimo provjeriti, a u trećem dijelu testa provjeravamo jesu li posljedice izvršavanja drugoga dijela testa u skladu s očekivanjima.
			
			Na slici 5.1 nalazi se primjer unit testa. Linije 380-399 su Arrange, linija 401 je Act, dok su linije 404-408 Assert dio testa.
			
			\begin{figure}[H]
				\begin{center}
					\includegraphics[width=13cm]{slike/junit_test.PNG}
				\end{center}
				\caption{Primjer JUnit testa}
				\label{fig:junit}
			\end{figure}
		
			
			Pošto je pisanje testova često i iscrpnije od pisanja implementacije (BandService ima 210 linija koda, dok BandServiceTest koji ni ne testira baš sve metode u njemu ima 558), za ovaj projekt nismo radili TDD (test driven development) već smo naknadno radili testove za postojeću implementaciju kako bismo potvrdili implementaciju i programski dokumentirali očekivano ponašanje.
			
			Uz dodatak BandService testovima postoji i nekoliko testova u EmailSenderTest koji pokazuju kako testirati komponentu čiju implementaciju ne znamo.
			
			
			Uz tridesetak JUnit testova čija je glavna prednost brzo izvođenje, napisali smo desetak integracijskih testova. Kao što smo već spomenuli, integracijski testovi slični su ručnom pregledavanju u Postmanu. Svaki integracijski test pokreće aplikaciju ispočetka tako da su mu stanje baze i aplikacije (kontekst) jednaki kao kad se pokrene aplikacija. U ovim testovima koristimo MockMvc kako bismo simulirali http request kojemu moramo odrediti metodu (GET, POST) te dodati pripadajuća zaglavlja. U ovom testu verificiramo je li ono što je vratila metoda jednako onome što se očekivalo (najčešće usporedbe DTO-ova).
			
			\begin{figure}[H]
				\begin{center}
					\includegraphics[width=17cm]{slike/integracijski_test.PNG}
				\end{center}
				\caption{Primjer integracijskog testa}
				\label{fig:inttest}
			\end{figure}
			
			Tehnike testiranja programske potpore nadahnute su knjigom Test-Driven Development Kenta Becka.
			
			
			
			\subsection{Ispitivanje sustava}
			
			 \textit{Potrebno je provesti i opisati ispitivanje sustava koristeći radni okvir Selenium\footnote{\url{https://www.seleniumhq.org/}}. Razraditi \textbf{minimalno 4 ispitna slučaja} u kojima će se ispitati redovni slučajevi, rubni uvjeti te poziv funkcionalnosti koja nije implementirana/izaziva pogrešku kako bi se vidjelo na koji način sustav reagira kada nešto nije u potpunosti ostvareno. Ispitni slučaj se treba sastojati od ulaza (npr. korisničko ime i lozinka), očekivanog izlaza ili rezultata, koraka ispitivanja i dobivenog izlaza ili rezultata.\\ }
			 
			 \textit{Izradu ispitnih slučajeva pomoću radnog okvira Selenium moguće je provesti pomoću jednog od sljedeća dva alata:}
			 \begin{itemize}
			 	\item \textit{dodatak za preglednik \textbf{Selenium IDE} - snimanje korisnikovih akcija radi automatskog ponavljanja ispita	}
			 	\item \textit{\textbf{Selenium WebDriver} - podrška za pisanje ispita u jezicima Java, C\#, PHP koristeći posebno programsko sučelje.}
			 \end{itemize}
		 	\textit{Detalji o korištenju alata Selenium bit će prikazani na posebnom predavanju tijekom semestra.}
			
			\eject 
		
		
		\section{Dijagram razmještaja}
			 
			 Dijagrami razmještaja opisuju topologiju sklopovlja i programsku potporu koja se koristi u implementaciji sustava u njegovom radnom okruženju. Sve komponente programske potpore smještene su na oblak platformu Heroku. Heroku kao platforma omogućuje programerima izvođenje i operiranje nad aplikacijama u oblaku. U ovome slučaju pozadinska i prednja aplikacija razmještene su na zasebene poslužitelje kao i baza podataka. Sustav je baziran na arhitekturi klijent - poslužitelj i komunikacija između njih odvija se HTTP protokolom. 
			 
			 	\begin{figure}[H]
			 	\begin{center}
			 		\includegraphics[width=15cm]{slike/deploy_fin.PNG}
			 	\end{center}
			 	\caption{Dijagram razmještaja}
			 	\label{fig:deploy_pic}
			 \end{figure}
			
			\eject 
		
		\section{Upute za puštanje u pogon}
		
			\textbf{\textit{dio 2. revizije}}\\
		
			 \textit{U ovom poglavlju potrebno je dati upute za puštanje u pogon (engl. deployment) ostvarene aplikacije. Na primjer, za web aplikacije, opisati postupak kojim se od izvornog kôda dolazi do potpuno postavljene baze podataka i poslužitelja koji odgovara na upite korisnika. Za mobilnu aplikaciju, postupak kojim se aplikacija izgradi, te postavi na neku od trgovina. Za stolnu (engl. desktop) aplikaciju, postupak kojim se aplikacija instalira na računalo. Ukoliko mobilne i stolne aplikacije komuniciraju s poslužiteljem i/ili bazom podataka, opisati i postupak njihovog postavljanja. Pri izradi uputa preporučuje se \textbf{naglasiti korake instalacije uporabom natuknica} te koristiti što je više moguće \textbf{slike ekrana} (engl. screenshots) kako bi upute bile jasne i jednostavne za slijediti.}
			
			
			 \textit{Dovršenu aplikaciju potrebno je pokrenuti na javno dostupnom poslužitelju. Studentima se preporuča korištenje neke od sljedećih besplatnih usluga: \href{https://aws.amazon.com/}{Amazon AWS}, \href{https://azure.microsoft.com/en-us/}{Microsoft Azure} ili \href{https://www.heroku.com/}{Heroku}. Mobilne aplikacije trebaju biti objavljene na F-Droid, Google Play ili Amazon App trgovini.}
			
			
			\eject 