\chapter{Opis projektnog zadatka}
		
		\textbf{\textit{dio 1. revizije}}\\
		
		\textit{Na osnovi projektnog zadatka detaljno opisati korisničke zahtjeve. Što jasnije opisati cilj projektnog zadatka, razraditi problematiku zadatka, dodati nove aspekte problema i potencijalnih rješenja. Očekuje se minimalno 3, a poželjno 4-5 stranica opisa.	Teme koje treba dodatno razraditi u ovom poglavlju su:}
		\begin{packed_item}
			\item \textit{potencijalna korist ovog projekta}
			\item \textit{postojeća slična rješenja (istražiti i ukratko opisati razlike u odnosu na zadani zadatak). Dodajte slike koja predočavaju slična rješenja.}
			\item \textit{skup korisnika koji bi mogao biti zainteresiran za ostvareno rješenje.}
			\item \textit{mogućnost prilagodbe rješenja }
			\item \textit{opseg projektnog zadatka}
			\item \textit{moguće nadogradnje projektnog zadatka}
		\end{packed_item}
		
		\textit{Za pomoć pogledati reference navedene u poglavlju „Popis literature“, a po potrebi konzultirati sadržaj na internetu koji nudi dobre smjernice u tom pogledu.}
		\eject
		
		Cilj ovog projekta je razviti programsku podršku za stvaranje web aplikacije Giger koja je namijenjena glazbenicima, ljudima kojima su potrebne usluge glazbenika (organizatori proslava, menadžeri) te ljudima koji su zainteresirani za obližnje događaje. To su ujedno i tri uloge unutar aplikacije vidljive korisnicima (Glazbenik, Organizator, Javnost). Uz te tri uloge, postoji i uloga Admin koja može ubaciti događaje koji će se prikažati u preporukama.
		
		Organizacija osobnog kalendara i dogovaranje termina koji zahtijevaju prisustvovanje više ljudi je težak zadatak, a s tim se problemom na gotovo dnevnoj razini susreću glazbenici kada dogovaraju nastupe. Isto tako, ljudi koji organiziraju razne proslave za koje trebaju glazbenike, kao i ljudi koji žele otići na neku živu svirku, ponekad ne znaju kakva je glazbena ponuda u njihovoj okolini, a ako su i čuli za neki događaj u blizini, ne postoji jedinstveno mjesto gdje mogu pročitati recenzije o glazbenicima, bendu ili organizatoru da budu sigurni u kvalitetu događaja.
		
		Giger je platforma koja rješava navedene probleme integrirajući kalendar glazbenika sa servisom namijenjenim za sastajanje bendova i organizatora nastupa. Česta je situacija da je jedan glazbenik član više bendova, pa tako dostupnost svakog benda ovisi o dostupnosti svih njegovih članova, a kalendar svakog glazbenika ujedinjuje sve obaveze iz bendova u kojima je član pa dostupnost benda postaje trivijalna informacija. Organizatori nastupa koji traže glazbenike imaju mogućnost filtrirati grupe prema vrsti glazbe, tipu nastupa, lokaciji i dr. Nakon što organizator dobije listu raspoloživih bendova, može pregledati njihove profile. Profil benda prikazuje osnovne informacije o bendu, razne novosti koje uređuju njegovi članovi, popis nadolazećih javnih nastupa te recenzije korisnika. Organizator ima mogućnost kontaktiranja benda putem poruka ugrađenih u aplikaciju. Imajući takav skup podataka, Giger svojim korisnicima, običnim ljudima željnih zabave, može preporučiti nadolazeće događaje u njihovoj blizini, a pošto su developeri koji razvijaju Giger ljudi od izvrsnog glazbenog ukusa, među preporukama nadolazećih evenata moći će se naći i događaji koji nisu ostvareni izravno putem aplikacije.
		
		\section{Slična rješenja problema}
		
		\subsection{Amy}
		
		Amy je mobilna aplikacija koja nakon registracije nudi različite opcije vrste korisnika (glazbenik, DJ i slično). Nakon odabira vrste korisnika nudi i unos instrumenata koje korisnik svira te odabir razine profesionalnosti na pojedinom instrumentu. Također, nudi i odabir glazbenih žanrova, kalendar s obavezama te pregledavanje glazbenika u blizini.  
		
		
	    \subsection{BandFriend}
		
		BandFriend je mobilna aplikacija koja prilikom registracije traži dosta informacija o korisniku. Nakon izrade profila, mogu se pretraživati novi glazbenici, glazbenici najsličniji trenutnom korisniku, po lokaciji i slično. Aplikacija nudi i razgovor porukama s drugim korisnicima.
		\section{Primjeri u LaTeXu}
		
		\textit{Ovo potpoglavlje izbrisati.}\\

		U nastavku se nalaze različiti primjeri kako koristiti osnovne funkcionalnosti LaTeXa koje su potrebne za izradu dokumentacije. Za dodatnu pomoć obratiti se asistentu na projektu ili potražiti upute na sljedećim web sjedištima:
		\begin{itemize}
			\item Upute za izradu diplomskog rada u LaTeXu - \url{https://www.fer.unizg.hr/_download/repository/LaTeX-upute.pdf}
			\item LaTeX projekt - \url{https://www.latex-project.org/help/}
			\item StackExchange za Tex - \url{https://tex.stackexchange.com/}\\
		
		\end{itemize} 	


		
		%Ovo poglavlje je potrebno prilikom predaje obrisati
		
		\underbar{podcrtani tekst}, 
		\textbf{podebljani tekst}, 
		\textit{nagnuti tekst}\\
		\normalsize primjer
		\large primjer
		\Large primjer
		\LARGE {primjer}
		\huge {primjer}
		\Huge primjer
		\normalsize
				
		\begin{packed_item}
			
			\item  primjer
			\item  primjer
			\item  primjer
			\item[] \begin{packed_enum}
				
				\item primjer
				\item primjer
			\end{packed_enum}
			
		\end{packed_item}
		
		\noindent primjer url-a: \url{https://www.fer.unizg.hr/predmet/opp/projekt}
		
		
		\begin{longtabu} to \textwidth {|X[8, l]|X[8, l]|X[16, l]|} %definicija sirine polja
			
			\hline \multicolumn{3}{|c|}{\textbf{naslov unutar tablice}}	 \\[3pt] \hline
			\endfirsthead
			
			\hline \multicolumn{3}{|c|}{\textbf{naslov unutar tablice}}	 \\[3pt] \hline
			\endhead
			
			\hline 
			\endlastfoot
			
			\rowcolor{LightGreen}IDKorisnik & INT	&  	Lorem ipsum dolor sit amet, consectetur adipiscing elit, sed do eiusmod  	\\ \hline
			korisnickoIme	& VARCHAR &   	\\ \hline 
			email & VARCHAR &   \\ \hline 
			ime & VARCHAR	&  		\\ \hline 
			\cellcolor{LightBlue} primjer	& VARCHAR &   	\\ \hline 
			
			
		\end{longtabu}
		

		\begin{table}[H]
			
			
			
			\begin{longtabu} to \textwidth {|X[8, l]|X[8, l]|X[16, l]|} %definicija sirine polja
				
				\hline 
				\endfirsthead
				
				\hline 
				\endhead
				
				\hline 
				\endlastfoot
				
				\rowcolor{LightGreen}IDKorisnik & INT	&  	Lorem ipsum dolor sit amet, consectetur adipiscing elit, sed do eiusmod  	\\ \hline
				korisnickoIme	& VARCHAR &   	\\ \hline 
				email & VARCHAR &   \\ \hline 
				ime & VARCHAR	&  		\\ \hline 
				\cellcolor{LightBlue} primjer	& VARCHAR &   	\\ \hline 
				
				
			\end{longtabu}
	
			\caption{\label{tab:referencatablica} Naslov ispod tablice.}
		\end{table}
		
		\begin{figure}[H]
			\includegraphics[scale=0.4]{slike/aktivnost.PNG}
			\centering
			\caption{Primjer slike s potpisom}
			\label{fig:promjene}
		\end{figure}
		
		\begin{figure}[H]
			\includegraphics[width=\linewidth]{slike/aktivnost.PNG}
			\caption{Primjer slike s potpisom 2}
			\label{fig:promjene2}
		\end{figure}
		
		
		
		\eject
		
	