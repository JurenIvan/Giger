\chapter{Opis projektnog zadatka}
			
		Cilj ovog projekta je razviti programsku podršku za stvaranje web aplikacije \textit{Giger} koja je namijenjena glazbenicima, ljudima kojima su potrebne usluge glazbenika (organizatori različitih proslava, menadžeri) te ljudima koji su zainteresirani za obližnje događaje. Na taj način na jednom mjestu se omogućava:
		
		\begin{packed_item}
			\item upoznavanje glazbenika i njihovo povezivanje u bendove
			\item olakšana komunikacija i organizacija unutar bendova
			\item promocija glazbenika
			\item dogovaranje nastupa s organizatorima koji jednostavno mogu pronaći prikladan bend za određeni događaj
			\item pregled zanimljivih nadolazećih događaja dostupan svima
			\item uvid u recenzije bendova/organizatora
		\end{packed_item}
	

		Organizacija osobnog kalendara i dogovaranje termina koji zahtijevaju prisustvovanje više ljudi je težak zadatak, a s tim se problemom na gotovo dnevnoj razini susreću glazbenici kada dogovaraju nastupe. Isto tako, ljudi koji organiziraju razne proslave za koje trebaju glazbenike, kao i ljudi koji žele otići na neku živu svirku, ponekad ne znaju kakva je glazbena ponuda u njihovoj okolini, a ako su i čuli za neki događaj u blizini, ne postoji jedinstveno mjesto gdje mogu pročitati recenzije o glazbenicima, bendu ili organizatoru kako bi bili sigurni u kvalitetu događaja.
		\textit{Giger} je platforma koja rješava navedene probleme integrirajući kalendar glazbenika sa servisom namijenjenim za sastajanje bendova i organizatora nastupa. 
		\\
		
		Česta je situacija da je jedan glazbenik član više bendova pa tako dostupnost svakog benda ovisi o dostupnosti svih njegovih članova.  Kalendar svakog glazbenika ujedinjuje sve obaveze iz bendova u kojima je član pa dostupnost benda postaje trivijalna informacija. 
		
		
		
		\section{Tijek i opseg aplikacije}
		
		Prilikom pokretanja aplikacije, neregistriranom korisniku prikazuju se opće informacije o javnim događajima te mu se nudi mogućnost prijavljivanja u sustav s postojećim računom (potrebno upisati email i lozinku) te kreiranje novog računa. Za stvaranje novog računa potrebni su: 
		
		\begin{packed_item}
			\item korisničko ime
			\item email adresa
			\item lozinka
		\end{packed_item}
	
		Registrirani korisnik može pregledati, mijenjati osobne podatke i izbrisati svoj korisnički račun. Takvom korisniku nudi se mogućnost pisanja recenzija te razmjenjivanja poruka s drugim korisnicima. On u svojim postavkama može postati glazbenik i/ili organizator.
		
		\textit{\underline{Glazbenik}} može odabrati instrumente koje svira, osnovati bend, pridružiti se postojećem, dodati članove u bend te uređivati svoj profil i kalendar. On na svojoj stranici može dodavati različite medije te se tako promovirati. 
		
		\textit{\underline{Organizator}} ima mogućnost kreiranja nastupa. On može filtrirati bendove prema vrsti glazbe, tipu nastupa, lokaciji i drugo. Nakon što dobije listu raspoloživih bendova, može pregledavati njihove profile. Profil benda prikazuje osnovne informacije o bendu, razne novosti koje uređuju njegovi članovi, popis nadolazećih javnih nastupa te recenzije korisnika. Organizator ima mogućnost kontaktiranja benda putem poruka ugrađenih u aplikaciju. Imajući takav skup podataka, \textit{Giger} svojim korisnicima, običnim ljudima željnim zabave, može preporučiti nadolazeće događaje u njihovoj blizini.
		
		Uz glazbenika i organizatora postoji i uloga \textit{\underline{administratora}} koji ima mogućnost uređivanja popisa ponuđenih instrumenata te blokiranja korisnika.
		\\
		
		Aplikacija će biti izvedena kao web aplikacija prilagođena (engl. responsive) mobilnom uređaju i podržavat će rad više paralelnih korisnika sa sučeljem koji je jednostavan za korištenje kako bi korisnici imali što bolje iskustvo.
		
		\subsection{Opcionalna proširenja aplikacije}
		
		\begin{packed_item}
			\item Bendovi mogu postaviti oglas da traže nove članove, na koje se glazbenici mogu javljati
			\item Moguće organizirati „Nasumična druženja“ da se hrpa glazbenika nađe i sviraju zajedno bez da su dio istog benda.
			\item Dodavanje komentara na objave
			\item Dodavanje novih funkcionalnosti za organizatore (mogućnost unajmljivanja ton majstora, fotografa itd.)
			\item Dodati „Podijeli na Facebook“ mogućnost za javne nastupe
		\end{packed_item}
		
		
		
		\section{Slična rješenja problema}
		
		\subsection{Amy}
		
		\textit{Amy} je mobilna aplikacija koja nakon registracije nudi različite opcije vrste korisnika (glazbenik, DJ i slično). Nakon odabira vrste korisnika nudi i unos instrumenata koje korisnik svira te odabir razine profesionalnosti na pojedinom instrumentu. Također, nudi i odabir glazbenih žanrova, kalendar s obavezama te pregledavanje glazbenika u blizini. Aplikacija ima jako puno potencijala, ali se često ruši što utječe na iskustvo korisnika. 
		\\
		
		\begin{figure}[H]
			\begin{center}
				\includegraphics[width=5cm]{slike/Amy.JPEG}
			\includegraphics[width=5cm]{slike/Amy2.JPEG}
			\end{center}
			\caption{Prikaz \textit{Amy} aplikacije}
			\label{fig:promjene2}
		\end{figure}
		
		
		
	    \subsection{BandFriend}
		
		\textit{BandFriend} je mobilna aplikacija koja prilikom registracije traži dosta informacija o korisniku. Nakon izrade profila, mogu se pretraživati novi glazbenici, glazbenici najsličniji trenutnom korisniku, najbliži po lokaciji i slično. Aplikacija nudi i razgovor porukama s drugim korisnicima. Neke korisnike tolika lista zahtjeva prilikom registracije može odbiti te će \textit{Giger} tražiti samo korisničko ime, email i lozinku, a kasnije svaki korisnik, ukoliko to želi, može dodati više informacija o sebi. Također, \textit{BandFriend} ne nudi kalendar s obavezama korisnika, stvaranje bendova ili događaja.
		\\
		
		\begin{figure}[H]
			\begin{center}
				\includegraphics[height=10cm]{slike/BandFriend.JPEG}
				\includegraphics[height=10cm]{slike/BandFriend2.JPEG}
			\end{center}
			\caption{Prikaz \textit{BandFriend} aplikacije}
			\label{fig:promjene3}
		\end{figure}

		
		\eject
		
	